\documentclass[letter,11pt,oneside,spanish]{article}

\usepackage[utf8x]{inputenc}
\usepackage[spanish]{babel}
\usepackage{graphicx}
\usepackage{float}
\usepackage{anysize}

\usepackage[
pdfauthor={Vladimir Cespedes Lopez},%
pdftitle={Ella},%
colorlinks,%
citecolor=black,%
filecolor=black,%
linkcolor=black,%
%urlcolor=black
pdftex]{hyperref}

\marginsize{2.5cm}{2.5cm}{2.5cm}{2.5cm}
\renewcommand{\baselinestretch}{1.5}

%opening
\title{\textbf{Ella: Buscador de niños perdidos}}
\author{Vladimir Cespedes Lopez\\
Perfil de Proyecto\\
Universidad Mayor de San Simón\\
\{vladycespedes@gmail.com\\}
\date{}
\begin{document}

\begin{titlepage}
\thispagestyle{empty}
\begin{center}
\large{\textsc{\bf Universidad Mayor De San Simón}}\\
\large{\textsc{\bf Facultad De Ciencias y Tecnología}}\\
\vspace{4.0cm}
\large{\bf Encontrar niños perdidos a través de \\
una aplicación móvil, usando geolocalización}\\
\vspace{1.0cm}
\small{Vladimir Cespedes Lopez}
~\\
\small{\today}
\end{center}
\end{titlepage}

\newpage
\tableofcontents

\newpage
\section{Antecedentes}
En los últimos años la perdida de niños va incrementando, de manera que las cifras de los 
casos de perdida sin resolver se hacen cada vez mas, es un problema que todos ven pero 
nadie ayuda a resolverlo.

Los centros comerciales, playas y los parques públicos son los sitios que 
concentran un mayor porcentaje de niños perdidos. Entre las madres que confiesan 
haber perdido a sus niños alguna vez, un 58\% declara haberlos perdido a en el 
centro comercial, un 16\% en la playa, un 15\% en un parque público y el 11\% 
restante se reparte entre sitios como parques temáticos, aeropuertos, 
espectáculos y conciertos, entre otros.

En la mayoría de los casos, fue la propia madre quién encontró a sus niños 
después de buscar enérgicamente a los pequeños y no fue necesaria la 
intervención de los Cuerpos de Seguridad. Aun así, durante la búsqueda, 
una de cada tres madres reconoce haber pedido ayuda a trabajadores locales 
y otras personas de la zona del incidente.

En la decada actual se tiene facil acceso a telefonos inteligentes y esto ira creciendo,
 gracias a esto, todos podemos estar comunicados, a partir de aquello, nace la idea de 
 ayudar a reducir el creciento de niños desaparecidos en ciertas condiciones.
 
Si bien el mayor culpable de este crecimiento es el descuido de los padres, podemos 
ayudar a reducir este, colaborando en la busqueda del niño, 

Actualmente las formas de encontrar que existen son fotos de los niños pegados en 
peajes, estaciones policiales, trancas, etc., también se ve tales fotos en el reverso de 
algunas facturas de empresas (como Elfec). Pero ninguna de ellas nos ayuda de manera
instantánea o de manera permanente.

También aplicaciones de mensajería instantánea como Whatsapp, facebook messenger, telegram  entre otros,
podrían ayudan actualmente a encontrar a los niños, enviando mensajes indicando datos
como el lugar de la perdida a todos los contactos posibles.

Debido a todo esto surgió la idea de realizar un proyecto que pueda ayudar a 
encontrarlos a través de una aplicación que informe a sus usuarios de la 
perdida tal que ellos puedan ayudar a encontrarlo y reportar del encuentro.

\section{Análisis del problema}
\subsection{Árbol del problema}
Véase el cuadro \ref{arboldeproyecto} en la página \pageref{arboldeproyecto}

\begin{figure}
    \centering
    \def\svgwidth{\columnwidth}
    \input{image.pdf_tex}
    \caption{Árbol de problemas}
	\label{arboldeproyecto}
\end{figure}

\subsection{Formulación del Problema}
Considerando los antecedentes mencionados anteriormente, se puede concluir que:

\begin{itemize}
\item Existen pocas opciones para el ayudar a encontrar niños perdidos.
\item Todas las soluciones actuales son artesanales y no son centralizadas.
\item El tiempo para encontrar a niños perdido es importante.
\end{itemize}

Por lo mencionado anteriormente se define el problema como:

\emph{Búsqueda  sosegada de niños perdidos reduce la probabilidad de encontrarlos.}


\section{Objetivo del proyecto}
\subsection{Objetivo general}
Acelerar la propagación de la búsqueda de niños perdidos mediante geolocalizacion 
para minimizar el tiempo para encontrarlos.

\subsection{Objetivos específicos}
\begin{itemize}
\item Agilizar la búsqueda del niño para minimizar el tiempo de desamparo.
\item Facilitar el acceso a la comunidad para acelerar el inicio de la búsqueda.
\item Mejorar los canales de comunicación entre la comunidad para facilitar la colaboración entre ellos.
\item Centralizar los reportes de niños encontrados para acelerar el encuentro de los afectados.
\end{itemize}

\section{Área de conocimiento}
\textbf{Área:} Ingeniería de Software
\textbf{Subárea:} Programación Web

\newpage

\section{Justificación}
Ya no sera necesario ir preguntando persona por persona, si alguien vio
algún niño perdido, el sistema se encargara de hacer llegar tal pregunta a 
todos los  usuarios que estén cerca del acontecimiento.

Ademas de poder llegar a todos los usuarios cercanos al incidente, se podrá 
ir propagando la información de la perdida en forma de anillo, tal que se pueda
llegar a todos los usuarios de manera incremental.

Esto es un paso para el avance tecnológico a nivel de comunidades.

\section{Alcance}

\section{Metodología}

\section{Bibliografia}


%\section{Planteamiento del Problema}
%En caso de perdida de un niño o niña, generalmente la forma de búsqueda es personal,
%esto quiere decir, ir preguntado persona por persona o en otros casos solo hacer una 
%busqueda visual, para poder encontrar al niño perdido. Considerando que el niño peligra
%mientras mas tiempo pase soló en lugares públicos, el tiempo juega un rol muy importante
%a la hora de encontrarlo. Entonces se hace notoria la falta de un espacio común de 
%ayuda que este orientado a niños perdidos.
%
%Ademas se ha notado que si bien las personas quieren ayudar, los canales de 
%comunicación para hacerlo son difíciles, y escasos, lo cual conlleva a crear 
%métodos creativos(tales como grupos en whatsapp, facebook u otras aplicaciones
%de mensajería instantánea), pero no globales.

\newpage


\textbf{Univ. Vladimir Cespedes Lopez}\\
\textbf{Nacionalidad:} boliviana\\
\textbf{Correo electrónico:} vladycespedes@gmail.com\\
Estudiante de la Carrera de Ingenieria de Sistemas (UMSS).\\
\end{document}
