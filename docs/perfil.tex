\documentclass[letter,11pt,oneside,spanish]{article}

\usepackage[utf8x]{inputenc}
\usepackage[spanish]{babel}
\addto{\captionsspanish}{%
\renewcommand{\bibname}{Trasferencias}
}
\usepackage{graphicx}
\usepackage{float}
\usepackage{anysize}
\usepackage{enumitem}
\usepackage{titlesec}

\usepackage[
pdfauthor={Vladimir Cespedes Lopez},%
pdftitle={Ella},%
colorlinks,%
citecolor=black,%
filecolor=black,%
linkcolor=black,%
%urlcolor=black
pdftex]{hyperref}

\marginsize{2.5cm}{2.5cm}{2.5cm}{2.5cm}
\renewcommand{\baselinestretch}{1.5}
\newcommand{\tabitem}{~~\llap{\textbullet}~~}

\renewcommand\refname{asdfr}

%opening
\title{\textbf{Ella: Buscador de niños perdidos}}
\author{Vladimir Cespedes Lopez\\
Perfil de Proyecto\\
Universidad Mayor de San Simón\\
\{vladycespedes@gmail.com\\}
\date{}
\begin{document}

\begin{titlepage}
\thispagestyle{empty}
\begin{center}
\large{\textsc{\bf Universidad Mayor De San Simón}}\\
\large{\textsc{\bf Facultad De Ciencias y Tecnología}}\\
\vspace{4.0cm}
\large{\bf Encontrar niños perdidos a través de \\
una aplicación móvil, usando geolocalización}\\
\vspace{1.0cm}
\small{Vladimir Cespedes Lopez}
~\\
\small{\today}
\end{center}
\end{titlepage}

\newpage
\tableofcontents

\newpage
\section{Antecedentes}
En los últimos años la perdida de niños va incrementando, de manera que las cifras de los 
casos de perdida sin resolver se hacen cada vez mas, es un problema que todos ven pero 
nadie ayuda a resolverlo.

Los centros comerciales, playas y los parques públicos son los sitios que 
concentran un mayor porcentaje de niños perdidos. Entre las madres que confiesan 
haber perdido a sus niños alguna vez, un 58\% declara haberlos perdido a en el 
centro comercial, un 16\% en la playa, un 15\% en un parque público y el 11\% 
restante se reparte entre sitios como parques temáticos, aeropuertos, 
espectáculos y conciertos, entre otros \cite{Delphi1}.

En la mayoría de los casos, fue la propia madre quién encontró a sus niños 
después de buscar enérgicamente a los pequeños y no fue necesaria la 
intervención de los Cuerpos de Seguridad. Aun así, durante la búsqueda, 
una de cada tres madres reconoce haber pedido ayuda a trabajadores locales 
y otras personas de la zona del incidente.

En la década actual se tiene fácil acceso a teléfonos inteligentes y esto ira creciendo,
 gracias a esto, todos podemos estar comunicados, a partir de aquello, nace la idea de 
 ayudar a reducir el crecimiento de niños desaparecidos en ciertas condiciones.
 
Si bien el mayor culpable de este crecimiento es el descuido de los padres, podemos 
ayudar a reducir éste, colaborando en la búsqueda del niño, 

Actualmente las formas de encontrar que existen son fotos de los niños pegados en 
peajes, estaciones policiales, trancas, etc., también se ve tales fotos en el reverso de 
algunas facturas de empresas (como Elfec). Pero ninguna de ellas nos ayuda de manera
instantánea o de manera permanente.

También aplicaciones de mensajería instantánea como Whatsapp, facebook messenger, telegram  entre otros,
podrían ayudan actualmente a encontrar a los niños, enviando mensajes indicando datos
como el lugar de la pérdida a todos los contactos posibles.

Debido a todo esto surgió la idea de realizar un proyecto que pueda ayudar a 
encontrarlos a través de una aplicación que informe a sus usuarios de la 
perdida tal que ellos puedan ayudar a encontrarlo y reportar del encuentro.

\section{Análisis del problema}
\subsection{Árbol del problema}
Véase el cuadro \ref{arboldeproyecto} en la página \pageref{arboldeproyecto}

\begin{figure}
    \centering
    \def\svgwidth{\columnwidth}
    \input{image.pdf_tex}
    \caption{Árbol de problemas}
	\label{arboldeproyecto}
\end{figure}

\subsection{Definición del Problema}

Búsqueda  sosegada de niños perdidos reduce la probabilidad de encontrarlos.


\section{Objetivo del proyecto}
\subsection{Objetivo general}

Propagar el reporte de niños perdidos a usuarios cercanos al lugar de la perdida mediante una aplicación móvil 
para localizar al niño de manera conjunta y rápida.

\subsection{Objetivos específicos}
\begin{itemize}
\item Determinar la plataforma de desarrollo y uso de la aplicación.
\item Diseñar la interfaz de usuario basada en la simplicidad que permita al usuario reportar la perdida de manera inmediata.
\item Permitir crear cuentas a través de la aplicación móvil asociadas a redes sociales.
\item Elaborar algoritmos para la propagación del reporte a los usuarios mas cercanos al incidente.
\item Validar el comportamiento de algoritmos aplicados al sistema.
\item Disponer de un servicio web de información geográfica \cite{Delphi2} para la ubicación de lugares.
\end{itemize}

\section{Área de conocimiento}
\textbf{Área:} Ingeniería de Software
\textbf{Subárea:} Programación Web

\section{Justificación}
Este proyecto sobre la búsqueda de niños perdidos bajo los siguientes criterios. Se reporta a tiempo la perdida y se tiene instalada la aplicación.
Aclarando estos criterios, podemos decir que, ya no sera necesario ir preguntando persona por persona, si alguien vio
algún niño perdido, el sistema se encargara de hacer llegar tal pregunta a 
todos los  usuarios que estén cerca del acontecimiento.

Ademas de poder llegar a todos los usuarios cercanos al incidente, se podrá 
ir propagando la información de la perdida en forma de anillo, tal que se pueda
llegar a todos los usuarios de manera incremental.

Esto es un paso para el avance tecnológico a nivel de comunidades.

\section{Alcance}
El desarrollo de este sistema considera toda la interacción entre distintas
instancias del sistema, es decir, en lo que respecta a autentificación, consumo y provisión de recursos y control de privilegios. Es necesario mencionar
también que escapan de las funciones de este sistema la interacción entre el
sistema desarrollado y otras redes sociales(a excepción de Facebook, que por este se hará el registro), sea para provisión o consumo de
recursos.

\section{Metodología}
Para el desarrollo del proyecto se hará uso de la metodología SCRUM, las fases y actividades a realizar son:

\begin{table}
\centering
\begin{tabular}{|p{3.5cm}|p{3.5cm}|p{3.5cm}|c|c|c|}
	\hline
	\multicolumn{6}{|c|}{Metodología del proyecto} \\ \hline
	Objetivo 	& Actividad 						& 	Resultado  	& Tiempo 	   & Fecha & Fecha   \\
	 				&										& 	esperado		& estimado  & Inicio 	& Fin   \\ \hline
	 Determinar la arquitectura e historias de usuario. & \begin{itemize}[leftmargin=*]
	 																					\item Generar Historias de usuario, priorizadas y estimadas.
																					 	\item Reparto de historias de usuario por iteración.
																					 	\item Estudio de la arquitectura.
																					\end{itemize}  	& \begin{itemize}[leftmargin=*]
	 																					\item Historias de usuario ordenadas.
																					 	\item Cantidad de historias de usuario determinada.
																					 	\item Arquitectura definida.
																					\end{itemize}  	& 10 días & 04-07-16 & 15-07-16										\\ 
	\hline %Fin del Product Backlog
	Facilitar el reporte inmediato de niños perdidos para minimizar el tiempo de desamparo. & \begin{itemize}[leftmargin=*]
	 																					\item Planificar Sprints(Sprint Planning).
																					 	\item Desarrollo de tareas.
																					 	\item Validar Sprints concluidos(Sprint Review).
																					\end{itemize}  	& \begin{itemize}[leftmargin=*]
	 																					\item Sprint Planificado.
																					 	\item Tareas terminadas.
																					 	\item Sprint Retrospective.
																					\end{itemize}  	& 10 días & 04-07-16 & 15-07-16										\\ 
	\hline %Fin del Primer Sprint
\end{tabular}
\end{table}
\begin{table}
\centering
\begin{tabular}{|p{3.5cm}|p{3.5cm}|p{3.5cm}|c|c|c|}
	\hline
	Agilizar la propagación del reporte a la comunidad para acelerar el inicio de la búsqueda en conjunto. & \begin{itemize}[leftmargin=*]
	 																					\item Planificar Sprints(Sprint Planning).
																					 	\item Desarrollo de tareas.
																					 	\item Validar Sprints concluidos(Sprint Review).
																					\end{itemize}  	& \begin{itemize}[leftmargin=*]
	 																					\item Sprint Planificado.
																					 	\item Tareas terminadas.
																					 	\item Sprint Retrospective.
																					\end{itemize}  	& 10 días & 04-07-16 & 15-07-16										\\ 
	\hline %Fin del Primer Sprint	
	Agilizar los procesos de publicación, y discriminación a colaboradores para llegar a los mas cercanos. & \begin{itemize}[leftmargin=*]
	 																					\item Planificar Sprints(Sprint Planning).
																					 	\item Desarrollo de tareas.
																					 	\item Validar Sprints concluidos(Sprint Review).
																					\end{itemize}  	& \begin{itemize}[leftmargin=*]
	 																					\item Sprint Planificado.
																					 	\item Tareas terminadas.
																					 	\item Sprint Retrospective.
																					\end{itemize}  	& 10 días & 04-07-16 & 15-07-16										\\ 
	\hline %Fin del Primer Sprint	
\end{tabular}
\end{table}
\begin{table}
\centering
\begin{tabular}{|p{3.5cm}|p{3.5cm}|p{3.5cm}|c|c|c|}
	\hline
	Centralizar los reportes de niños encontrados para acelerar el encuentro de los afectados. & \begin{itemize}[leftmargin=*]
	 																					\item Planificar Sprints(Sprint Planning).
																					 	\item Desarrollo de tareas.
																					 	\item Validar Sprints concluidos(Sprint Review).
																					\end{itemize}  	& \begin{itemize}[leftmargin=*]
	 																					\item Sprint Planificado.
																					 	\item Tareas terminadas.
																					 	\item Sprint Retrospective.
																					\end{itemize}  	& 10 días & 04-07-16 & 15-07-16										\\ 
	\hline %Fin del Primer Sprint	
	Puesta en marcha y pruebas de carga & \begin{itemize}[leftmargin=*]
	 																					\item Despliegue de el sistema.
																					 	\item Pruebas de carga.
																					 	\item Generación de "scripts" de recuperación de sistema.
																					\end{itemize}  	& \begin{itemize}[leftmargin=*]
	 																					\item Sistema en producción.
																					 	\item Balanceo de Carga.
																					 	\item Pruebas de "scripts" generados.
																					\end{itemize}  	& 10 días & 04-07-16 & 15-07-16										\\ 
	\hline %Fin del Primer Sprint
\end{tabular}
\end{table}

\newpage

\section{Bibliografia}

\begingroup
\titleformat*{\section}{\normalfont}
\begin{thebibliography}{depth}

\bibitem{Delphi1} Los lugares donde es más frecuente perder a tu hijo.\\
Extraído el 28 de Mayo del 2016, de\\
http://www.abc.es/familia-padres-hijos/20150619/abci-perder-201506181233.html

\bibitem{Delphi2} Sistema de información geográfica.\\
Extraído el 04 de Junio del 2016, de\\
https://es.wikipedia.org/wiki/Sistema\_de\_información\_geográfica

\end{thebibliography}
\endgroup

%\section{Planteamiento del Problema}
%En caso de perdida de un niño o niña, generalmente la forma de búsqueda es personal,
%esto quiere decir, ir preguntado persona por persona o en otros casos solo hacer una 
%busqueda visual, para poder encontrar al niño perdido. Considerando que el niño peligra
%mientras mas tiempo pase soló en lugares públicos, el tiempo juega un rol muy importante
%a la hora de encontrarlo. Entonces se hace notoria la falta de un espacio común de 
%ayuda que este orientado a niños perdidos.
%
%Ademas se ha notado que si bien las personas quieren ayudar, los canales de 
%comunicación para hacerlo son difíciles, y escasos, lo cual conlleva a crear 
%métodos creativos(tales como grupos en whatsapp, facebook u otras aplicaciones
%de mensajería instantánea), pero no globales.

\end{document}
